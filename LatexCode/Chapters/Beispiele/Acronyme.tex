%************************************************
%*  Abk�rzungen *********************************
%************************************************

\section{Abk�rzungen}

Um Abk�rzungen zu verwenden, muss �ber \lstinline|\usepackage{acronym}| das ben�tigte Package geladen werden.

Ein kleiner Test k�nnte so aussehen: \\

Dies ist eine \ac{Abk.}, die beim ersten Aufruf mit Erkl�rung und bei allen weiteren Malen nur als \ac{Abk.} dargestellt wird. Auch der Plural von \aclp{Abk.} kann definiert und abgerufen werden. Mit anderen Befehlen kann man auch die Erkl�rung mitliefern \acf{z.B.} so. Wer nur die Abk�rzung mag, f�gt sie \acs{z.B.} so ein. Mit \lstinline|\acused{Abk.}| wird die \acl{Abk.} als genutzt markiert und taucht im Folgenden nur noch in seiner Kurzform auf.
 
 Ganz am Ende der Arbeit wird das Abk�rzungsverzeichnis eingef�hrt (auch als eigenes Chapter m�glich): 

\begin{acronym}
 	\acro{Abk.}{Abk�rzung}
 	\acroplural{Abk.}[Abk.]{Abk�rzungen}
 	\acro{z.B.}{zum Beispiel}
\end{acronym}

Weitere Informationen sind im \textit{Acronym-Manual} zu finden.