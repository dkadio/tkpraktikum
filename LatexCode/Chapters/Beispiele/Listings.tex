%*******************************
% 			Listings 		   *
%*******************************

\section{Quellcode einbinden}
Das Package \textit{lstlisting} erm�glicht es, Quellcode ansprechend in das Dokument einzubinden. Man kann Quellcode einzeilig einbinden, 
mittels \lstinline{\lstinline|Quellcode|}. Dabei ist darauf zu achten, dass der Befehl einmal mit \{ \} und einmal mit | | aufgerufen werden kann, je nachdem
, welche Zeichen im angegebenen Quelltext genutzt werden. 
Es ist auch m�glich eine eigene Umgebung f�r Quelltext zu schaffen:

\begin{lstlisting}[caption=Erstes Listing,style=Java]
private Umgebung(int i, int k)
{
	System.out.println("Eine Funktion mit " + i + "und" + k ".");
}
\end{lstlisting}  

Wer Quelltext aus externen Dateien einbinden m�chte, geht wie folgt vor:

\lstinputlisting
[caption={Externer Quellcode},style=Java]
{Examples/Code.java}

Wie genau der Quellcode formatiert und gef�rbt ist, ist in \textit{htwsaar.i.mst.config.tex} hinterlegt, wobei f� verschiedene Sprachen auch eigene Styles angelegt werden
k�nnen (hier z.B. f�r Java).