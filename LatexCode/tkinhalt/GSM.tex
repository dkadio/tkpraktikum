%=========================================
% 	   Einleitung     		 =
%=========================================

\chapter{GSM Versuch}
\section{Allgemeine Beschreibung der Versuche}
Im folgenden handelt es sich um ein Test-Versuch im Global System for Mobile Communications.
Es wird an zwei Baugleichen Systemen gearbeitet die jeweils eine Universal Software Radio Peripheral anbieten (USRP). Das System l�uft mit dem Programm OpenBTS und implementiert einen GSM-Protokollstack von Layer 1-3 und terminiert die h�heren Schichten. Mit dem System ist es m�glich die meisten GSM-Signale abzufangen und mit zu schneiden.
Ziel des Versuches ist es die Packetdaten via Wireshark, in dem GSM-Netz von einem Anruf auf den Echo-Server sowie eine SMS an die 411 mi dem Text "info", mit zu schneiden und zu analysieren. Um sich mit den Komponenten und GSM vertraut zu machen werden zu Anfang einige Visuelle und Informative Versuche ausgef�hrt wie z.b. das Visualisieren von Frequenzen und das erarbeiten der mathematischen Zusammen h�nge der Frequenzen.

%Kurze Einleitung ins Thema
\subsection{Versuchsaufbau}

Bestandteile des Versuchsaufbaus sind zwei baugleiche Open Base Transceiver Station Systeme bestehend aus Computer und der USRP. Die USRP ist f�r den Empfang der Funksignale notwendig. OpenBTS l�uft in unserem Fall auf einem Rechner mit Ubuntu als Betriebssystem und  besteht aus mehren Programmen. Das System modifiziert den gew�hnlichen GSM-Netzaufbau. USRP, SDR und OpenBTS �bernehmen die Aufgaben von dem Base Transceiver Station und dme Base Station Controller, die Aufgabe des Mobile Switching Center wird von dem Asterisk �bernommen und verbindet das Netz mit dem IP-Backbone. SDR steht f�r Software Defined Radio und stellt Signalverarbeitungsbibliotheken zur verf�gung. 
Es wird ein Mobiltelefon das bereits im Netz registriert ist bereit gestellt, es ist jedoch eben so gut m�glich sich mit einem anderen GSM-f�higen Telefon in dem Netz anzumelden. Auf dem OpenBTS system laufen verschiedene Dienste wie etwa der echo dienst der unter der Nummer 2600 bzw eine reply-Dients f�r SMS unter der 411.

%womit und in welcher Anordnung wurde protokolliert
\section{Visualisieren von Frequenzen}
Im folgenden Versuch wird mit Hilfe zweier Tools Frequenzen empfangen und diese Visualisiert. Das Tool kal scannt alle empfangbaren Frequenzen ab, zeigt deren Downlink sowie ARFCN und die st�rke des empfangenden Signals an. ARFCN steht f�r Absolute Radio Channel Number durch die man die Down- sowie Uplinkfrequenzen berechnen kann. Im GSM 1800 sind die ARFCN von 512 bis 885 zugeordnet. Die geringste Downlinkfrequenz bei GSM 1800 ist 1805,2 MHz. Passend dazu sind die Uplinkfrequenzen in einem Abstand von 95 Mhz, beginnend bei 1710,2 bis 1784,8 Mhz. Jedes Down und Uplink-Paar wird durch die ARFCN gekennzeichnet. Durch das Tool baudline ist es m�glich die empfangenen Frequenzen zeitlich zu betrachten. Um die Benutzung zu vereinfachen benutzen wir dbusrp. 



Hier sollten noch die berechnungen stehen und vlt die tabelle zu den Provider frequenzen.

\subsection{Versuchsdurchf�hrung}
%was wurde protokolliert

\subsection{Versuchsziel}
%warum wurde protokoliert


\section{Anruf an die 2600}
\subsection{Versuchsdurchf�hrung}
%was wurde protokolliert
\subsection{Versuchsziel}
%warum wurde protokoliert
\section{Beschreibung der Verschiedenen Messungen und Ergebnisdarstellung}
\section{Diskussion der Messergebnisse und Ausarbeiten der Aufgaben}
%dies beinhaltet auch die Nacharbeitung von Themen, die im Versuch als bisher nicht bekannt erkannt wurden.


\section{Senden einer SMS an die 411}
\subsection{Versuchsdurchf�hrung}
%was wurde protokolliert
\subsection{Versuchsziel}
%warum wurde protokoliert
\section{Beschreibung der Verschiedenen Messungen und Ergebnisdarstellung}
\section{Diskussion der Messergebnisse und Ausarbeiten der Aufgaben}
%dies beinhaltet auch die Nacharbeitung von Themen, die im Versuch als bisher nicht bekannt erkannt wurden.





