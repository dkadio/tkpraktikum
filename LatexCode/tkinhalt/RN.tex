%=========================================
% 	   Einleitung     		 =
%=========================================

\chapter{RN Versuch}

\begin{table}
\begin{tabular}{|*{4}{c|}}
\hline
Netzadresse & BC & Hosts & Subnet Mask \\
\hline
\hline
192.168.0.1 & 192.168.1.127 & 192.168.1.1 - 192.168.1.126 & 255.255.255.128 \\
\hline
 192.168.0.128 & 192.168.1.191 & 192.168.1.129 - 192.168.1.190 & 255.255.255.192 \\
\hline
192.168.0.192 & 192.168.1.199 & 192.168.1.193 - 192.168.1.198 & 255.255.255.248 \\

\hline
 & &  FE0/0 & 172.18.0.1 &  255.255.0.0\\
\hline
\end{tabular}
\caption{Adresstabelle des Netzwerks}
\label{adresstabelle}
\end{table}


