\section{Dateien einbinden}

Damit man nicht alle Einstellungen, Optionen, Packages und Texte, Abbildungen etc. in einer Datei unterbringen muss, werden
zwei Befehle bereitgestellt, um externe \textit{*.tex}-Dateien einzubinden: \lstinline|\include{PFAD}| und \lstinline|\input{PFAD}|.
Mit dem erstem Befehl wird eine neue Seite angelegt, danach kommen die Inhalte aus der angegebenen Datei; mit dem zweiten
Befehl wird keine neue Seite angelegt -- der Inhalt der angegebenen Datei wird direkt an die betroffene Stelle eingef�gt.

\textbf{Wichtig:} Der \textit{Pfad} wird sinnigerweise \textit{relativ} angegeben, wobei als Stammverzeichnis jenes Verzeichnis
angesehen wird, in dem die \textit{*.tex}-Datei mit der \textit{Document}-Umgebung abgelegt ist (in diesem Fall ist es 
\textit{htwsaar-i-mst-config.tex}).